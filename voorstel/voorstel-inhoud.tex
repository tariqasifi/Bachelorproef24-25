%---------- Inleiding ---------------------------------------------------------

% TODO: Is dit voorstel gebaseerd op een paper van Research Methods die je
% vorig jaar hebt ingediend? Heb je daarbij eventueel samengewerkt met een
% andere student?
% Zo ja, haal dan de tekst hieronder uit commentaar en pas aan.

%\paragraph{Opmerking}

% Dit voorstel is gebaseerd op het onderzoeksvoorstel dat werd geschreven in het
% kader van het vak Research Methods dat ik (vorig/dit) academiejaar heb
% uitgewerkt (met medesturent VOORNAAM NAAM als mede-auteur).
% 

\section{Inleiding}%
\label{sec:inleiding}

Man-in-the-Middle (MitM)-aanvallen vormen een grote bedreiging voor de veiligheid en vertrouwelijkheid van gegevens binnen netwerken. Bij dit type aanval plaatst een aanvaller zich onopgemerkt tussen twee communicerende partijen, waardoor hij toegang krijgt tot gevoelige informatie die vervolgens onderschept, gemanipuleerd of misbruikt kan worden. Deze aanvallen maken gebruik van kwetsbaarheden in netwerken, onveilige verbindingen en onvoldoende beveiligingsmaatregelen, wat de veiligheid en integriteit van netwerken in gevaar brengt. Dit kan niet alleen leiden tot beveiligingsrisico’s, maar ook tot verlies van vertrouwen en integriteit in netwerkomgevingen.
Vanwege de groeiende dreiging van MitM-aanvallen is er behoefte aan een diepgaand onderzoek naar effectieve preventiemethoden. Dit onderzoek heeft als doel verschillende soorten MitM-aanvallen vast te stellen, te onderzoeken en te vergelijken, en de maatregelen te identificeren die organisaties kunnen nemen om deze risico’s te verkleinen. Op deze manier biedt het onderzoek organisaties handvatten om hun netwerken beter te beschermen en de kans op MitM-aanvallen te minimaliseren.

Daarnaast bevat het onderzoek een Proof of Concept (PoC), waarbij minstens één methode ter voorkoming van MitM wordt gedemonstreerd en getest. De bevindingen kunnen organisaties helpen bij het implementeren van effectieve MitM-beveiligingsmaatregelen te implementeren, afgestemd op diverse netwerkomgevingen en met oog voor schaalbaarheid en kosten.

De centrale onderzoeksvragen die in dit onderzoek aan bod komen, zijn als volgt:

\begin{itemize}
  \item Welke soorten Man-in-the-Middle-aanvallen komen het vaakst voor, en waarin verschillen ze?
  
  \item Welke beveiligingsmaatregelen worden momenteel genomen om deze aanvallen te voorkomen?
  
  \item Hoe effectief zijn verschillende beveiligingsmaatregelen in het waarborgen van de integriteit en vertrouwelijkheid van gegevens?
  \item Welke invloed hebben deze veiligheidsmaatregelen op de prestaties van het netwerk, en hoe wordt dit gemeten?
  
  \item Zijn organisaties zich voldoende bewust van de risico’s van MitM-aanvallen en de noodzakelijke preventieve maatregelen?
\end{itemize}

%---------- Stand van zaken ---------------------------------------------------

\section{Literatuurstudie}%
\label{sec:literatuurstudie}
Voor dit onderzoek heb ik bestaande studies geraadpleegd om inzicht te krijgen in Man-in-the-Middle (MitM)-aanvallen en de beschikbare beveiligingsmaatregelen. Dit overzicht vormt de basis voor een diepgaander onderzoek naar methoden om MitM-aanvallen in netwerken effectief te voorkomen en te detecteren.

MitM-aanvallen vormen een aanzienlijke bedreiging voor de vertrouwelijkheid en integriteit van digitale communicatie. Aanvallers kunnen ongemerkt gegevens tussen partijen onderscheppen en manipuleren door kwetsbaarheden in netwerkprotocollen en communicatiekanalen uit te benutten 
~\autocite{ELRAWY2023}. Deze aanvallen komen vaak voor door een gebrek aan voldoende sterke beveiligingsmaatregelen in netwerken  ~\autocite{HALGAMUGE2025}. Onderzoek wijst op het belang van sterke maatregelen, zoals encryptie en authenticatie, om deze risico’s te verkleinen en de betrouwbaarheid van netwerken te waarborgen ~\autocite{ALIYU201824}.
\\
\\
\textbf{Veelvoorkomende Soorten MitM-aanvallen}
\begin{enumerate}
  \item \textbf{Openbare Wi-Fi-aanvallen:}  
  Man-in-the-Middle (MitM)-aanvallen op openbare Wi-Fi-netwerken gebeuren wanneer een aanvaller communicatie onderschept en manipuleert. Veelgebruikte technieken zijn Multi-Channel MitM (MC-MitM), nep-Wi-Fi-netwerken (Evil Twin) en Channel switch (CSA). Kwetsbaarheden ontstaan door ontbrekende updates en zwakke beveiligingsstandaarden. Bescherming is mogelijk met beveiligde Wi-Fi, regelmatige updates en versleutelde websites (HTTPS) ~\autocite{THANKAPPAN2022}.
  
  \item \textbf{Phishing-gebaseerde MitM:}  
  Via phishing misleiden aanvallers slachtoffers door hen te laten klikken op schadelijke links of malware te installeren, vaak via valse e-mails of frauduleuze websites. Hierdoor krijgen aanvallers toegang tot gevoelige gegevens door slachtoffers te misleiden met geloofwaardige en professionele inhoud ~\autocite{Nmachi2023}.
  
  \item \textbf{Domeinspoofing-aanval met HTTPS:}  
  Aanvallers registreren domeinen die sterke gelijkenissen vertonen met echte websites en beveiligen deze met valse HTTPS-certificaten. Dit creëert een vals gevoel van veiligheid bij gebruikers, waardoor zij in de verleiding komen om persoonlijke gegevens in te voeren, omdat het domein lijkt te vertrouwen op een veilige verbinding ​~\autocite{gangan2015}.
  
  \item \textbf{SSL-manipulatie:}  
  Aanvallers genereren frauduleuze SSL-certificaten om gebruikers te misleiden en naar kwaadwillige websites te leiden die eruitzien als echte, beveiligde sites. Hierdoor kunnen zij gevoelige informatie onderscheppen zonder dat de gebruikers zich hiervan bewust zijn ~\autocite{gangan2015}.
  
  \item \textbf{Sessiekaping (Session Hijacking):}  
  Aanvallers stelen sessiecookies om toegang te krijgen tot gebruikersaccounts, zonder dat het slachtoffer zich opnieuw hoeft aan te melden. Dit gebeurt door sessiecookies te capteren tijdens een actieve sessie, waardoor de aanvaller volledige controle over het account verwerft ~\autocite{gangan2015}.
  
  \item \textbf{E-mailovername:}  
  Aanvallers onderscheppen en manipuleren e-mails door zich voor te doen als vertrouwde organisaties of personen. Dit gebeurt vaak in financiële transacties en bij gevoelige bedrijfsinformatie, waarbij de aanvaller zich voordoet als een betrouwbare afzender om betalingen of vertrouwelijke gegevens te verkrijgen~\autocite{Nmachi2023}.
  
  \item \textbf{ARP-spoofing:}  
  Aanvallers versturen vervalste ARP-antwoorden binnen een lokaal netwerk, waardoor de koppeling tussen IP- en MAC-adressen in de ARP-cache van doelcomputers wordt gemanipuleerd. Hierdoor wordt netwerkverkeer naar het apparaat van de aanvaller geleid, wat hen de mogelijkheid biedt om gegevens te onderscheppen, te wijzigen of extra pakketten in te voegen ~\autocite{arslan2017}.
\end{enumerate}



\textbf{Verschillen Tussen Soorten MitM-aanvallen}

\begin{description}
  \item[\textbf{Techniek:}]  
  Sommige aanvallen, zoals ARP-spoofing, richten zich specifiek op netwerkprotocollen, terwijl anderen, zoals phishing, afhankelijk zijn van social engineering.
  
  \item[\textbf{Doelwit:}]  
  Openbare Wi-Fi-aanvallen richten zich vaak op individuele gebruikers, terwijl e-mailovername en sessiekaping meer gericht zijn op bedrijven of financiële instellingen.
  
  \item[\textbf{Complexiteit:}]  
  HTTPS-spoofing en SSL-manipulatie vereisen technische kennis en voorbereiding, zoals het registreren van domeinen, terwijl phishing eenvoudiger kan worden uitgevoerd.
  
  \item[\textbf{Impact:}]  
  Sessiekaping kan directe toegang geven tot accounts, terwijl phishing-gebaseerde aanvallen vaak dienen als opstap naar verdere uitbuiting.
\end{description}

Verschillende studies beschrijven specifieke methoden om MitM-aanvallen te voorkomen. Het combineren van Advanced Encryption Standard (AES) met Diffie-Hellman-sleuteldistributie wordt gezien als een effectieve oplossing om zowel de vertrouwelijkheid als de integriteit van gegevens te waarborgen~\autocite{ELRAWY2023}. Bovendien voegt deze aanpak slechts minimale overhead toe aan netwerkprestaties, waardoor het geschikt blijft voor netwerken met beperkte middelen ~\autocite{HALGAMUGE2025}. Een gelaagde beveiligingsaanpak met de integratie van Intrusion Detection Systems (IDS) en Intrusion Prevention Systems (IPS) kan bijdragen aan het vroegtijdig detecteren en voorkomen van aanvallen   ~\autocite{ANKARI2022}.

Daarnaast geven studies inzicht in de uitdagingen van MitM-beveiliging binnen Internet of Things (IoT)-omgevingen. IoT-apparaten zijn vaak kwetsbaar door hun beperkte rekenkracht en energiecapaciteit, waardoor traditionele beveiligingsmethoden minder effectief zijn. Lichtgewicht cryptografische oplossingen zijn noodzakelijk om deze apparaten effectief tegen MitM-aanvallen te beschermen zonder hun prestaties te beperken ~\autocite{ANKARI2022}. Tot slot benadrukken onderzoeken dat dynamische beveiligingsmodellen, zoals adaptieve randbeveiliging, essentieel zijn om IoT-apparaten te beschermen tegen evoluerende cyberdreigingen ~\autocite{HALGAMUGE2025}.

Deze studie legt een basis voor verder onderzoek naar MitM-aanvallen en beschermingsstrategieën binnen mijn bachelorproject. De verschillende onderdelen – zoals aanvalstypen, beveiligingsmaatregelen en uitdagingen in  netwerkomgevingen – dienen als vertrekpunt voor het ontwikkelen en analyseren van effectieve preventie- en detectiemethoden.


% Voor literatuurverwijzingen zijn er twee belangrijke commando's:
% \autocite{KEY} => (Auteur, jaartal) Gebruik dit als de naam van de auteur
%   geen onderdeel is van de zin.
% \textcite{KEY} => Auteur (jaartal)  Gebruik dit als de auteursnaam wel een
%   functie heeft in de zin (bv. ``Uit onderzoek door Doll & Hill (1954) bleek
%   ...'')

%---------- Methodologie ------------------------------------------------------
\section{Methodologie}%
\label{sec:methodologie}
Dit onderzoek volgt een gestructureerde aanpak, verdeeld in vier opeenvolgende fasen om systematisch inzicht te verkrijgen in Man-in-the-Middle (MitM)-aanvallen en de meest geschikte beveiligingsmaatregelen voor diverse netwerkomgevingen.
\vspace{0.5cm}
In de eerste fase voeren we een grondig literatuuronderzoek uit naar verschillende soorten MitM-aanvallen en de beveiligingsmethoden die momenteel beschikbaar zijn. We gebruiken wetenschappelijke artikelen, rapporten en casestudies om inzicht te krijgen in aanvalstechnieken zoals ARP-spoofing, DNS-spoofing en SSL-stripping. Dit onderzoek leidt tot een overzicht van aanvalstypen en mogelijke verdedigingsstrategieën en vormt de theoretische basis voor verdere analyse en praktische toepassing.
\vspace{0.5cm}

Op basis van de literatuur selecteren we relevante beveiligingsmaatregelen, zoals encryptie, authenticatie en detectiesystemen, die effectief kunnen zijn tegen MitM-aanvallen. Elke maatregel wordt beoordeeld op haar theoretische effectiviteit, haalbaarheid en geschiktheid voor specifieke netwerkomgevingen, waaronder IoT-apparaten. Deze analyse helpt ons om maatregelen te selecteren voor praktische evaluatie in de volgende fase.
\vspace{0.5cm}
In deze fase ontwikkelen we een Proof of Concept (PoC) voor één of meer geselecteerde beveiligingsmaatregelen en testen deze in een gesimuleerde netwerkomgeving. Hierin voeren we MitM-aanvalsscenario's uit om de effectiviteit en prestaties van de maatregelen te evalueren. De focus ligt op meetbare resultaten zoals gegevensintegriteit, vertrouwelijkheid en eventuele impact op netwerkprestaties. Deze testen bieden praktische inzichten in de haalbaarheid en kosten van de maatregelen.


Op basis van de testresultaten worden de beveiligingsmaatregelen geëvalueerd. Er wordt een implementatieadvies opgesteld voor organisaties, met aanbevelingen over de inzet van MitM-beveiliging in verschillende netwerkconfiguraties, inclusief IoT-omgevingen. Dit advies bevat praktische overwegingen voor schaalbaarheid en kosten, zodat organisaties goed geïnformeerde beslissingen kunnen nemen.
\vspace{0.5cm}

%---------- Verwachte resultaten ----------------------------------------------
\section{Verwacht resultaat}%
\label{sec:verwachte_resultaten}

Het onderzoek zal naar verwachting een overzicht bieden van de meest voorkomende Man-in-the-Middle (MitM)-aanvallen en laten zien hoe bestaande beveiligingsmaatregelen kunnen helpen deze te voorkomen. Door middel van een Proof of Concept wordt gedemonstreerd hoe een specifieke beveiligingsmethode effectief kan worden toegepast in een gesimuleerde omgeving. Het onderzoek zal daarnaast aanbevelingen geven voor organisaties over het verbeteren van hun netwerkbeveiliging om MitM-aanvallen beter te voorkomen en gegevensintegriteit en vertrouwelijkheid te waarborgen.