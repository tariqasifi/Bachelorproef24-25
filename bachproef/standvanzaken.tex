\chapter{\IfLanguageName{dutch}{Stand van zaken}{State of the art}}%
\label{ch:stand-van-zaken}

% Tip: Begin elk hoofdstuk met een paragraaf inleiding die beschrijft hoe
% dit hoofdstuk past binnen het geheel van de bachelorproef. Geef in het
% bijzonder aan wat de link is met het vorige en volgende hoofdstuk.
De stand van zaken biedt een overzicht van literatuur en theoretische inzichten die nodig zijn om de vergelijking tussen Cisco Packet Tracer en GNS3 te kaderen. We beginnen met een toelichting op het verschil tussen netwerksimulatie en netwerkemulatie (~\ref{sec:netwerksimulatie-vs-netwerkemulatie}), omdat deze begrippen belangrijk zijn om te begrijpen hoe Packet Tracer en GNS3 werken.

\vspace{0.3cm}

 Vervolgens gaan we dieper in op Cisco Packet Tracer (~\ref{sec:Cisco Packet Tracer}) en GNS3(~\ref{sec:GNS3}): we behandelen hun ontstaan, functionaliteiten, beperkingen en hoe ze worden gebruikt binnen de opleiding Netwerkbeheer. Tot slot worden in paragraaf (\ref{sec:Alternatieve tools: EVE-NG, Cisco VIRL en andere}) kort enkele alternatieve tools besproken, waaronder EVE-NG en Cisco VIRL, om een volledig beeld te geven van de beschikbare simulatie- en emulatiesoftware.
% Pas na deze inleidende paragraaf komt de eerste sectiehoofding.
\section{\IfLanguageName{dutch}{Netwerksimulatie vs. netwerkemulatie}{Network Simulation vs. Network Emulation}}%
\label{sec:netwerksimulatie-vs-netwerkemulatie}

De termen simulatie en emulatie worden in de informatica vaak door elkaar gehaald, maar er is wel degelijk een belangrijk onderscheid tussen beide, vooral binnen de context van virtuele netwerken.

\vspace{0.2cm}

Bij netwerksimulatie wordt het gedrag van netwerkapparaten of protocollen virtueel voorgesteld met behulp van vereenvoudigde softwaremodellen. Deze modellen tonen hoe het netwerk zich zou gedragen, maar voeren de interne processen van echte hardware, zoals een router, niet uit \autocite{GOMEZ2023}. Alleen de communicatie tussen apparaten wordt weergegeven; de interne verwerking van gegevens, zoals die binnen fysieke hardware plaatsvindt, blijft buiten beschouwing. Met andere woorden, een simulatie creëert de indruk van een functionerend netwerk, maar de onderliggende processen die bij fysieke apparaten plaatsvinden, worden niet uitgevoerd.

\vspace{0.2cm}

Netwerkemulatie gaat een stap verder dan simulatie. In plaats van enkel het externe gedrag weer te geven, voert emulatie ook de interne processen van apparaten of systemen daadwerkelijk uit. Het echte besturingssysteem of de firmware van een netwerkapparaat wordt daarbij geladen en uitgevoerd binnen een virtuele omgeving. Die omgeving zorgt ervoor dat het besturingssysteem denkt dat het op echte hardware draait, waardoor het zich bijna identiek gedraagt als in de werkelijkheid \autocite{asee_peer_2016}.

\vspace{0.2cm}

Een belangrijk voordeel hiervan is dat virtuele toestellen geen onderscheid maken tussen virtuele en fysieke hardware. Dankzij deze realistische werking kunnen geëmuleerde netwerken rechtstreeks worden gekoppeld aan fysieke netwerken of het internet. Routers en switches gedragen zich daarbij zoals fysieke toestellen, met ondersteuning voor dezelfde functies en protocollen \autocite{asee_peer_2016}.


\begin{table}[H]
    \centering
    \caption{Vergelijking tussen netwerksimulatie en -emulatie.}
    \label{tab:sim_vs_emul}
    \begin{tabularx}{\textwidth}{|l|X|X|}
        \hline
        \textbf{Kenmerk} & \textbf{Simulatie (Packet Tracer)} & \textbf{Emulatie (GNS3)} \\
        \hline
        \textbf{Werking} & Simuleert het gedrag van netwerkapparatuur zonder de echte software of het besturingssysteem van de apparaten te gebruiken. & Draait de echte firmware of het besturingssysteem virtueel en functioneert volledig zoals het fysieke apparaat. \\
        \hline
        \textbf{Bewustzijn van omgeving} & Weet dat het een simulatie is (vereenvoudigd gedrag). & Denkt dat het echt is (virtueel apparaat merkt emulatie niet). \\
        \hline
        \textbf{Nauwkeurigheid} & Beperkt tot gesimuleerde functies; niet alle eigenschappen van het echte apparaat zijn aanwezig. & Functioneert hetzelfde als een echt toestel en ondersteunt alle mogelijkheden. \\
        \hline
        \textbf{Resource-gebruik} & Efficiënt: werkt met een vereenvoudigde versie van een besturingssysteem, zonder een volledig besturingssysteem op elk toestel te draaien. & Zwaarder: elke node gebruikt CPU/RAM zoals een echt toestel. \\
        \hline
        \textbf{Gebruik in onderwijs} & Ideaal voor basisconcepten, visualisatie en eenvoudige labs. & Geschikt voor geavanceerde labs, realistische tests en integratie met echte netwerken. \\
        \hline

        
    \end{tabularx}
\noindent\footnotesize{\autocite{asee_peer_2016, GOMEZ2023}}
\end{table}


Simulators en emulators vullen elkaar aan: simulators zijn geschikt voor het aanleren van basisconcepten op een gebruiksvriendelijke manier, terwijl emulators belangrijk zijn om realistische configuraties en commando’s te testen zoals op echte netwerkapparatuur.


\section{\IfLanguageName{dutch}{Cisco Packet Tracer}{Cisco Packet Tracer}}
\label{sec:Cisco Packet Tracer}

Cisco Packet Tracer (PT) is een uitgebreide netwerksimulatorsoftware ontwikkeld door Cisco Systems, bedoeld om studenten en cursisten op een toegankelijke manier netwerkconfiguraties te laten oefenen in een virtuele omgeving. De software maakt deel uit van de lesmaterialen van de Cisco Networking Academy en speelt wereldwijd een belangrijke rol in opleidingen tot Cisco Certified Network Associate (CCNA). Packet Tracer is specifiek ontworpen voor onderwijsdoeleinden en stelt studenten in staat om op een veilige manier praktische netwerkvaardigheden te ontwikkelen ~\autocite{netacad2025}.



\vspace{0.3cm}



Packet Tracer biedt een virtuele omgeving waarin studenten computernetwerken kunnen ontwerpen, configureren en troubleshooten zonder fysieke hardware. De software stelt gebruikers in staat om realistische netwerktopologieën te creëren. Routers, switches, eindapparaten en andere componenten kunnen visueel worden geplaatst en verbonden, waarbij netwerkinstellingen en protocollen geconfigureerd kunnen worden alsof er met echte apparatuur wordt gewerkt~\autocite{netacad2025}.



\vspace{0.5cm}

\vspace{0.5cm}

\vspace{0.5cm}



\subsection{\IfLanguageName{dutch}{Functionaliteiten}{Functionalities}}%
\label{sec:Functionaliteiten}

Packet Tracer biedt een uitgebreide virtuele omgeving voor het veilig configureren en simuleren van netwerkarchitecturen. De software ondersteunt een breed scala aan gesimuleerde netwerkapparaten en netwerkprotocollen. Studenten kunnen aan de slag met een brede reeks netwerkapparaten, waaronder Catalyst-switches (2960, 3560), routers (1841, 2911, 4331), netwerkkaarten, hubs, access points en eindapparaten zoals pc’s, servers en IoT-sensoren. Op protocolniveau biedt Packet Tracer uitgebreide ondersteuning voor de configuratie van routeringsprotocollen zoals RIP, OSPF en EIGRP, en voor het opzetten van netwerksegmentatie door middel van VLAN’s en inter-VLAN routing. Daarnaast kunnen belangrijke netwerktechnieken zoals ARP, het Spanning Tree Protocol (STP) en EtherChannel worden toegepast om redundantie, schaalbaarheid en efficiënte bandbreedtebenutting binnen netwerken te simuleren ~\autocite{CiscoPacketTracerFAQ}.

\vspace{0.3cm}

Ook geavanceerdere configuraties, zoals het opzetten van DHCP-servers, DNS-diensten, NAT (Network Address Translation) en ACL’s (Access Control Lists), kunnen binnen de omgeving worden ingericht en getest ~\autocite{CiscoPacketTracerFAQ}.

\vspace{0.3cm}

Een belangrijk functioneel voordeel van Packet Tracer is dat configuratiewijzigingen in de \textit{real-time mode} onmiddellijk zichtbaar zijn, waardoor gebruikers direct de impact van hun configuratie kunnen zien ~\autocite{Kuzmenko2016}. Dit zorgt ervoor dat gebruikers het netwerkgedrag beter begrijpen en hun probleemoplossende vaardigheden verder ontwikkelen. Daarnaast biedt de \textit{simulation mode} de mogelijkheid om datapakketten stap voor stap te analyseren en protocollen zoals ICMP, TCP en routingupdates te volgen. Hierdoor kunnen gebruikers niet alleen problemen opsporen en oplossen, maar ook beter begrijpen hoe gegevens en protocollen zich binnen een netwerk verplaatsen.


\vspace{0.3cm}


Naast de uitgebreide configuratiemogelijkheden biedt Packet Tracer ook een multi-user functie, waarmee meerdere gebruikers tegelijkertijd kunnen samenwerken aan één gezamenlijke netwerkconfiguratie ~\autocite{Smith2010}. Deze mogelijkheid maakt het eenvoudig om projecten en labo-oefeningen uit te voeren, waarbij meerdere gebruikers samen aan één netwerkproject werken.

\vspace{0.3cm}

Daarnaast ondersteunt Packet Tracer de simulatie van IoT-apparaten, zoals slimme sensoren, actuatoren en bewakingscamera’s ~\autocite{thera2020}. Gebruikers kunnen deze apparaten niet alleen verbinden met hun netwerk, maar ook eenvoudige toepassingen programmeren, bijvoorbeeld via Python-scripts. Deze functionaliteit breidt de mogelijkheden van Packet Tracer aanzienlijk uit, doordat gebruikers niet langer beperkt zijn tot traditionele netwerkconfiguraties, maar ook leren hoe ze IoT-apparaten kunnen configureren en beheren.

\vspace{0.3cm}

Een ander belangrijk onderdeel van Packet Tracer is de ingebouwde \textit{Activity Wizard} en de toetsingsfunctie. Hiermee kunnen instructeurs scenario’s en labo-oefeningen maken waarbij studenten bepaalde configuraties moeten uitvoeren ~\autocite{asee_peer_2016}. De software controleert de oplossingen automatisch op basis van vooraf ingestelde regels, zodat studenten direct feedback krijgen en zowel zij als hun docenten kunnen zien of de configuraties correct zijn of fouten bevatten. Hierdoor ondersteunt Packet Tracer zowel zelfstandig leren als begeleide trainingen in het bouwen en beheren van netwerken ~\autocite{asee_peer_2016}.


\subsection{\IfLanguageName{dutch}{Voordelen in het onderwijs}{Benefits in education
}}
\label{sec:Voordelen in het onderwijs}
Cisco Packet Tracer biedt verschillende voordelen voor educatief gebruik. De software is volledig gratis te gebruiken. Het enige wat nodig is, is een gratis account bij de Cisco Networking Academy om toegang te krijgen tot de software. Packet Tracer is compatibel met verschillende besturingssystemen, waaronder Windows, macOS en Linux. De interface is gebruiksvriendelijk en vereist weinig voorkennis, wat de tool toegankelijk maakt voor beginners zonder ervaring met de Cisco CLI. Daarnaast bevat Packet Tracer een breed scala aan voorbeeldlabs en tutorials die gebruikers stap voor stap begeleiden bij het configureren van netwerkinstellingen~\autocite{packetTracerResources2025}.

\vspace{0.3cm}

Onderzoek van \textcite{noor2018} toont aan dat studenten die gebruikmaakten van Packet Tracer aanzienlijke verbeteringen vertoonden in hun vermogen om netwerktopologieën te ontwerpen en netwerkproblemen op te lossen. Andere studies bevestigen de waarde van Packet Tracer als leerhulpmiddel, vooral wanneer het gebruik ervan wordt gecombineerd met echte hardware om de overgang van simulatie naar praktijkervaring te ondersteunen \autocite{runtuwene2024}.

\vspace{0.3cm}

Daarnaast beschouwt Cisco Packet Tracer als een ideaal platform voor studenten die beginnen met het opbouwen van basiskennis en het opdoen van vaardigheden binnen het domein van netwerken. Dankzij de integratie in het officiële NetAcad-curriculum maakt de software wereldwijd deel uit van het standaardaanbod binnen netwerkgerichte opleidingen ~\autocite{packetTracerResources2025}.



\subsection{\IfLanguageName{dutch}{Gebruik in het onderwijs}{Gebruik in het onderwijs}}%
\label{sec:Gebruik in het onderwijs}

Cisco Packet Tracer is ontwikkeld met het oog op netwerkopleidingen en wordt vaak al vanaf de eerste lessen ingezet om studenten op een toegankelijke manier kennis te laten maken met netwerkinfrastructuur. Dankzij de gebruiksvriendelijke interface kunnen beginnende studenten zich focussen op concepten zoals IP-adressering, routeringstabellen en netwerkstructuren, zonder dat zij meteen diepgaand vertrouwd hoeven te zijn met command-line syntax. Inmiddels is het gebruik van Packet Tracer breed verspreid, ook buiten de NetAcad: vele universiteiten en hogescholen zetten het programma in voor vakken als computernetwerken, netwerkbeheer en cybersecurity ~\autocite{Allison2022}.

\vspace{0.3cm}

Na verloop van tijd, naarmate de kennis en vaardigheden van studenten groeien, leren zij ook werken met de Cisco CLI binnen Packet Tracer. De ingebouwde CLI-simulatie voor routers en switches vertoont sterke gelijkenissen met de structuur van echte Cisco IOS-apparatuur. Hierdoor kunnen studenten realistische commando’s en configuraties oefenen, wat hen optimaal voorbereidt op het gebruik van fysieke hardware of professionele emulatoren ~\autocite{Allison2022}.

\vspace{0.3cm}

Zoals eerder aangehaald, ondersteunt Packet Tracer naast de leermogelijkheden ook de evaluatie. Via de ingebouwde assessment-functionaliteiten kunnen docenten opdrachten ontwerpen waarbij studenten een netwerkconfiguratie moeten maken, die automatisch wordt nagekeken op correctheid. Een voorbeeld hiervan is de opleiding Systeem- en Netwerkbeheer aan HOGENT, waar studenten elk een gepersonaliseerde netwerksimulatie krijgen als praktijkexamen. Deze aanpak bevordert zelfstandig werken, maakt foutenanalyse eenvoudiger en beperkt fraude door unieke opdrachten per student ~\autocite{asee_peer_2016}.

\vspace{0.3cm}

Onderzoek toont bovendien aan dat Packet Tracer helpt om de kloof tussen theorie en praktijk te verkleinen. Studenten leren beter door zelf dingen te proberen, wat goed past bij actief leren ~\autocite{Noor2018}. Ook bleek dat studenten meer betrokken zijn en beter worden in routing wanneer ze oefenen met Packet Tracer ~\autocite{Noor2018}.

\vspace{0.3cm}

Een groot voordeel van Packet Tracer is dat studenten veilig kunnen oefenen. Ze kunnen zonder risico experimenteren met ACL’s, firewalls en zelfs cyberaanvallen ~\autocite{Allison2022}. Zo leren ze makkelijker, durven ze meer fouten te maken en begrijpen ze de leerstof beter ~\autocite{Allison2022}.

\vspace{0.3cm}

Cisco werkt Packet Tracer regelmatig bij met nieuwe technologieën zoals IoT en netwerkautomatisering. Studenten blijven zo op de hoogte van moderne ontwikkelingen binnen hetzelfde programma ~\autocite{CiscoPacketTracerFAQ}.

\vspace{0.3cm}

Ook verhoogt Packet Tracer de motivatie en leerresultaten. Studenten die ermee werkten, scoorden beter op testen, begrepen de leerstof sneller en zagen beter hoe alles in netwerken samenhangt ~\autocite{Javid2014}. Docenten kunnen bovendien makkelijker zien hoe ver studenten staan en fouten snel verbeteren ~\autocite{Javid2014}.

\vspace{0.3cm}

Ten slotte biedt Packet Tracer ook voordelen op financieel vlak: scholen besparen geld omdat ze minder dure apparatuur nodig hebben. Zelfs scholen met weinig middelen kunnen goed netwerkonderwijs aanbieden. Een lokaal met alleen computers is vaak al genoeg ~\autocite{Makasiranondh2010}.



\subsection{\IfLanguageName{dutch}{Beperkingen}{Beperkingen
}}
\label{sec:Beperkingen}

Ondanks de bovengenoemde voordelen, wijzen wetenschappelijke bronnen ook op enkele beperkingen en aandachtspunten bij het gebruik van Cisco Packet Tracer in het onderwijs. Allereerst blijft Packet Tracer een simulatieomgeving die het gedrag van netwerken op vereenvoudigde wijze weergeeft. In tegenstelling tot emulatoren zoals GNS3, maakt Packet Tracer geen gebruik van het echte besturingssysteem van Cisco-apparaten, zoals Cisco IOS. Hierdoor ontbreken veel functies of zijn slechts beperkt beschikbaar. Geavanceerde mogelijkheden zoals BGP-configuraties, MPLS, uitgebreide QoS-instellingen en specifieke varianten van het Spanning Tree Protocol zijn niet beschikbaar of slechts gedeeltelijk functioneel ~\autocite{Kuzmenko2016}. Ook de command-line interface (CLI) in Packet Tracer is beperkt tot de basis die nodig is voor het CCNA-niveau. Daardoor kunnen complexere configuraties en realistische scenario’s niet volledig worden uitgevoerd, zoals dat wel mogelijk is in omgevingen die werken met een echte IOS-image ~\autocite{Mwansa2024}.

\vspace{0.3cm}

Daarnaast is Packet Tracer volledig gericht op Cisco-apparatuur en biedt geen ondersteuning voor netwerkhardware van andere fabrikanten zoals Juniper, MikroTik of HP. Ook open-source netwerkoplossingen zoals pfSense, VyOS of Linux-gebaseerde routers worden niet ondersteund ~\autocite{Kuzmenko2016}. Hierdoor kunnen studenten geen ervaring opdoen met systemen buiten het Cisco-ecosysteem. Deze beperking maakt het moeilijk om studenten vertrouwd te maken met omgevingen waarin verschillende soorten apparatuur samen worden gebruikt, zoals vaak voorkomt in de praktijk. Voor opleidingen die inzetten op open standn of leverancier-onafhankelijk netwerkbeheer, kan dit ertoe leiden dat studenten niet volledig voorbereid zijn op de variatie en complexiteit van echte IT-omgevingen.

\vspace{0.3cm}

Een andere beperking is dat simulatie per definitie vereenvoudigd is. Packet Tracer biedt geen volledige emulatie van hardwarecomponenten: belangrijke fysieke factoren zoals werkelijke datasnelheden, CPU-belasting, kabelkwaliteit, latency of poortdefecten worden niet weergegeven ~\autocite{Hashimi2017}. Hierdoor kunnen studenten een onrealistisch beeld krijgen van de complexiteit van netwerkbeheer: in de simulator verlopen configuraties vaak soepel, terwijl men in de praktijk te maken krijgt met vertragingen, compatibiliteitsproblemen en hardwarefouten. Dit risico wordt bevestigd in verschillende onderzoeken. Zo merkten Elias en Mohamad Ali ~\autocite{Elias2014} op dat studenten die voornamelijk met Packet Tracer hadden geoefend, tegen onverwachte problemen aanliepen toen zij overstapten naar echte netwerkapparatuur. Ook Syahrizad en Zamzuri ~\autocite{Elias2014} concludeerden dat het curriculum dat volledig op simulatie gebaseerd was, onvoldoende was om studenten volledig voor te bereiden op real-world situaties. Zij raden daarom een hybride aanpak aan, waarbij simulaties worden gecombineerd met fysieke labs.

\vspace{0.3cm}

Hoewel veel universiteiten en hogescholen beschikken over fysieke netwerkapparatuur, lukt het in de praktijk niet altijd om alle studenten volledig de kans te geven om elk labo-onderdeel uit te voeren. Beperkingen zoals beperkte toegangstijd, een tekort aan toestellen of grote studentengroepen kunnen ervoor zorgen dat niet elke student evenveel praktische ervaring kan opdoen ~\autocite{inproceedings}.




\section{\IfLanguageName{dutch}{GNS3}{GNS3}}%
\label{sec:GNS3}

GNS3 (Graphical Network Simulator 3) is een open-source netwerkemulator, ontwikkeld in 2008 om netwerkprofessionals te ondersteunen bij de voorbereiding op certificatie-examens zoals Cisco CCNA en CCNP. In tegenstelling tot eenvoudige simulators zoals Cisco Packet Tracer, emuleert GNS3 echte netwerkapparatuur door gebruik te maken van officiële IOS-images en andere apparaatsoftware. Hierdoor draait een virtuele router in GNS3 dezelfde software als een fysieke router, waardoor het netwerkgedrag en de functionaliteit sterk overeenkomen met de praktijk ~\autocite{Kuzmenko2016}.

\vspace{0.3cm}

GNS3 ondersteunt veel verschillende netwerkapparaten en protocollen. Gebruikers kunnen werken met diverse Cisco-apparaten, zoals routers met IOS en Cisco ASA-firewalls, maar ook met apparatuur van andere leveranciers zoals Juniper (JunOS) en open-source routers zoals pfSense of VyOS ~\autocite{Amrizal2022}. Door gebruik te maken van officiële IOS-images en andere systeemsoftware zijn vrijwel alle commando’s en netwerkfuncties beschikbaar die men ook op echte hardware zou verwachten. GNS3 maakt het bovendien mogelijk om geavanceerde functies te gebruiken die in eenvoudigere simulaties vaak ontbreken ~\autocite{gns3_docs2025}.

\vspace{0.3cm}

GNS3 maakt het mogelijk om zowel eenvoudige als complexe netwerken te bouwen. Gebruikers kunnen uitgebreide netwerktopologieën ontwerpen met meerdere routers, switches, firewalls en eindapparaten, zonder dat daar fysieke hardware voor nodig is. Zelfs complexe bedrijfsnetwerken of examenlabs, zoals die voor Cisco’s CCNP-certificering, kunnen volledig virtueel worden opgezet ~\autocite{Gil2015}.

\vspace{0.3cm}

Daarnaast ondersteunt GNS3 ook oudere en minder gangbare technologieën die in andere simulatieplatformen vaak niet beschikbaar zijn. Zo kunnen bijvoorbeeld Frame Relay- en ATM-netwerken worden gesimuleerd voor educatieve toepassingen. Via modules zoals de NM-16ESW of virtuele Cisco IOSvL2-images is het mogelijk om Ethernet-switches te gebruiken. Dit maakt het eenvoudiger om netwerkconcepten zoals VLAN-configuraties en andere switchfunctionaliteiten toe te passen en te analyseren~\autocite{gns3_doc2025}.

\vspace{0.3cm}

Verder biedt GNS3 de mogelijkheid om virtuele netwerken te verbinden met fysieke infrastructuren. Zo kan een virtuele router gekoppeld worden aan een echt campusnetwerk of het internet, waardoor hybride scenario’s ontstaan waarin virtuele en fysieke omgevingen gecombineerd worden ~\autocite{gns3_doc2025}.

\vspace{0.3cm}

Tot slot beschikt GNS3 standaard over integratie met analysetools zoals Wireshark. Elke verbinding in de netwerktopologie kan worden gemonitord via packet capturing, waardoor studenten netwerkverkeer in \textit{realtime} kunnen analyseren en zo beter inzicht krijgen in de werking van protocollen ~\autocite{Golightly2023}.



\vspace{0.3cm}

\subsection{\IfLanguageName{dutch}{Functionaliteiten}{Functionalities}}%
\label{sec:Functionaliteiten}

GNS3 biedt, dankzij zijn uitgebreide mogelijkheden, een aanzienlijke meerwaarde voor netwerkopleidingen en systeembeheer in het hoger onderwijs. Ten eerste benadert een GNS3-lab de realiteit van een fysiek netwerk zeer nauwkeurig. Studenten voeren configuraties uit op een emulatie van echte Cisco- of Juniper-software, waarbij zij exact dezelfde commando’s gebruiken en dezelfde reacties krijgen als op een fysieke router of switch. Dit realisme versterkt de leerervaring: studenten leren werken met de echte CLI en functies van netwerkapparatuur, in plaats van met een vereenvoudigde simulatieomgeving ~\autocite{Kuzmenko2016}.

\vspace{0.3cm}

Onderzoek in academische context bevestigt dat GNS3 hierdoor realistischere netwerktopologieën ondersteunt dan traditionele simulators ~\autocite{bakni2019}. Concreet toonden onderzoekers aan dat GNS3 in staat is om volledige Cisco IOS-functionaliteit te draaien, waardoor diepgaandere netwerkexperimenten mogelijk worden dan bijvoorbeeld met Packet Tracer ~\autocite{bakni2019}.

\vspace{0.3cm}

Een tweede belangrijk voordeel is dat GNS3 studenten voorbereidt op professionele certificeringen en op de praktijk. Voor het instapniveau (CCNA) raadt Cisco Networking Academy vaak Packet Tracer aan, omdat deze leeromgeving laagdrempelig is ~\autocite{gns3_docs2025}. Voor gevorderde certificeringen zoals CCNP en zelfs CCIE is Packet Tracer echter onvoldoende, aangezien niet alle protocollen of commando’s worden ondersteund ~\autocite{gns3_docs2025}. GNS3 biedt hiervoor een oplossing: het werd oorspronkelijk ontwikkeld door oprichter Jeremy Grossman met als doel studenten te ondersteunen bij hun voorbereiding op het CCNP-examen ~\autocite{gns3_docs2025}.

\vspace{0.3cm}

Dankzij GNS3 kunnen studenten trainen in een bijna echte labomgeving, zonder dat dure fysieke hardware nodig is ~\autocite{gns3_docs2025}. Hierdoor wordt het voor hogescholen haalbaar om complexe scenario’s aan te bieden die anders enkel in dure fysieke labs of officiële Cisco-omgevingen mogelijk zijn. Bovendien dekt GNS3 vrijwel het volledige CCNA- en CCNP-curriculum qua functionaliteit, waardoor studenten bij het afleggen van examens al vertrouwd zijn met de juiste commando’s en het gedrag van professionele netwerkapparatuur.

\vspace{0.3cm}

GNS3 ondersteunt volgens de literatuur ook zelfstandig leren en experimenteren. Doordat de software gratis en open-source is en aansluit bij de professionele praktijk, kunnen studenten ook buiten de lesmomenten aan de slag. Ze bouwen thuis netwerkopstellingen, maken fouten en lossen deze zelf op via troubleshooting, met ondersteuning van de grote online community. Deze leerbenadering sluit aan bij constructivistische onderwijsmethoden, waarin actief ontdekken en probleemoplossend denken centraal staan. Studenten kunnen bovendien oefenen zonder risico op schade aan apparatuur, wat in fysieke labs buiten de geplande lessen vaak niet haalbaar is ~\autocite{Sari2018}.

\vspace{0.3cm}

Ten slotte biedt GNS3 diverse praktische voordelen ten opzichte van fysieke netwerklabs. Omdat alles virtueel gebeurt, bestaat er geen risico op schade aan dure netwerkapparatuur bij fouten. Eén krachtige computer is voldoende om tientallen virtuele routers tegelijk te draaien ~\autocite{Golightly2023}. Dit maakt het platform niet alleen aantrekkelijk voor instellingen met een beperkt budget, maar ook geschikt voor afstandsonderwijs, waar toegang tot fysieke infrastructuur vaak ontbreekt. Virtuele labs kunnen eenvoudig worden herstart, waardoor studenten oefeningen kunnen herhalen, feedback verwerken en hun vaardigheden geleidelijk verbeteren.

\subsection{\IfLanguageName{dutch}{Voordelen in het onderwijs}{Benefits in education
}}

GNS3 wordt in het hoger onderwijs vaak gebruikt als vervanging of aanvulling op fysieke netwerklabs. Doordat het echte netwerkbesturingssystemen emuleert, kunnen studenten realistische configuraties oefenen zonder dat dure apparatuur nodig is. Uit onderzoek blijkt dat GNS3 goed werkt om netwerkvaardigheden aan te leren, zelfs bij grote studentengroepen of beperkte beschikbaarheid van labmateriaal ~\autocite{Amrizal2022}. De grafische gebruikersinterface maakt het opzetten van netwerktopologieën toegankelijker, al vraagt de tool enige technische voorkennis. In tegenstelling tot tools zoals Cisco Packet Tracer moeten gebruikers zelf besturingssystemen voorzien en werken met een echte CLI ~\autocite{Sari2018}. Dit vraagt wat meer voorbereiding, maar studenten doen waardevolle ervaring op met realistische netwerkcommando’s ~\autocite{Amrizal2022}.

\vspace{0.3cm}

GNS3 ondersteunt zelfstandig leren en is ook geschikt voor thuisgebruik en afstandsonderwijs. Studenten kunnen het programma op hun eigen laptop installeren en op hun eigen tempo oefenen buiten de les. Uit onderzoek blijkt dat praktijkoefeningen met GNS3 ook op afstand goed werken~\autocite{Sari2018}, waardoor meer studenten toegang krijgen tot netwerktraining.

\vspace{0.3cm}

GNS3 wordt ook ingezet voor evaluaties en praktijkopdrachten. Docenten kunnen virtuele labs maken waarin studenten veilig en zelfstandig werken, zonder elkaar te storen in een gedeelde omgeving. Hoewel GNS3 geen ingebouwde automatische beoordelingsfuncties biedt, implementeren veel onderwijsinstellingen handmatige evaluatiemethoden of aangepaste scripts om de leerresultaten van studenten te beoordelen. ~\autocite{pythonautomation}

\vspace{0.3cm}

GNS3 ondersteunt daarnaast het werken met multivendor-omgevingen en biedt ondersteuning voor netwerkautomatisering, containers en SDN ~\autocite{Kuzmenko2016}. Deze mogelijkheden maken GNS3 geschikt voor netwerkonderwijs dat sterk gericht is op praktijk en realistische werksituaties.



\subsection{\IfLanguageName{dutch}{Beperkingen}{Beperkingen
}}
\label{sec:Beperkingen}

Daartegenover staan ook enkele nadelen aan het gebruik van GNS3, vooral op het vlak van gebruiksgemak en systeemvereisten. Ten eerste is GNS3 bekend om zijn hoge hardwarevereisten bij het draaien van complexe labs. Omdat elke router in feite een volledig besturingssysteem laadt, vraagt dit veel CPU en RAM. Zo kan één enkele Cisco IOSv-routerimage ongeveer 512 MB RAM en een deel van een CPU-core gebruiken ~\autocite{cml_faq2025}. Een lab met tien routers, vier switches en enkele clients kan dus al snel meerdere gigabytes aan geheugen en aanzienlijke verwerkingskracht vereisen, vooral tijdens het opstarten (het laden van IOS-images vraagt veel CPU).

\vspace{0.3cm}

Ter vergelijking: Packet Tracer verbruikt weinig systeembronnen omdat de volledige toepassing licht is opgebouwd. De virtuele apparaten gebruiken geen afzonderlijk RAM-geheugen zoals bij echte emulatie. GNS3 daarentegen vereist meer rekenkracht: wie met meerdere toestellen tegelijk wil werken, heeft een krachtige pc of server nodig, bij voorkeur met een recente CPU die virtualisatie ondersteunt en voldoende geheugen. Daardoor is GNS3 minder geschikt om op eender welk studententoestel te gebruiken, zeker als dat niet aan de minimumvereisten voldoet. In een klaslokaal is het dan aangewezen om gebruik te maken van performante pc’s of een gedeelde server ~\autocite{Kuzmenko2016}.

\vspace{0.3cm}

Een tweede beperking is de complexiteit van installatie en configuratie. In tegenstelling tot Packet Tracer, dat een standalone applicatie is, vereist GNS3 vaak het installeren en configureren van meerdere componenten (de GNS3 GUI, een GNS3 VM in VirtualBox/VMware voor betere prestaties, en het importeren van afzonderlijke apparaat-images). Volgens de officiële handleiding is GNS3 geen alles-in-één pakket. Gebruikers moeten de benodigde componenten afzonderlijk installeren en configureren, wat afhankelijk is van lokale instellingen op de computer, zoals firewallregels en gebruikersrechten ~\autocite{Golightly2023}.

\vspace{0.3cm}

Voor studenten (en zelfs docenten) met beperkte IT-ervaring kan dit een drempel vormen: er gaat kostbare tijd naar het opzetten van de omgeving vóór men aan de eigenlijke netwerkopdrachten toekomt \autocite{Amrizal2022}. Daarnaast moeten Cisco IOS-images of andere appliance-images extern verkregen worden, vaak onder licentie. GNS3 levert om juridische redenen geen Cisco-software mee; de gebruiker moet zelf een IOS-image importeren, bijvoorbeeld door die te downloaden via Cisco’s website (vereist meestal academische toegang of een servicecontract) of door een officiële Cisco VIRL-licentie (tegenwoordig CML-Personal) aan te schaffen ~\autocite{gns3_ios_images}. Dit vormt een organisatorische uitdaging in een onderwijssituatie: de hogeschool moet zorgen dat er een legale en eenvoudige manier is voor studenten om aan de benodigde images te komen. Sommige opleidingen lossen dit op door vooraf geconfigureerde GNS3-projecten te verspreiden waarin de node-definities (zonder de IOS zelf) al klaarstaan, of door een centrale repository te voorzien met images onder academische licentie.~\autocite{gns3_docs2025}

\vspace{0.3cm}

Ondanks de vele mogelijkheden vereist GNS3 aanzienlijk meer voorbereiding in vergelijking met bijvoorbeeld Cisco Packet Tracer, waarbij studenten na installatie meteen kunnen starten. Bij GNS3 zijn extra stappen nodig, zoals het opzetten van de virtuele machine en het configureren van netwerkverbindingen. Dit kan leiden tot technische problemen, zoals verbindingsfouten door foutieve VirtualBox-instellingen. Begeleiding bij installatie en troubleshooting is daarom vaak noodzakelijk en moet structureel opgenomen worden in het curriculum ~\autocite{gns3_docs2025}.

\vspace{0.3cm}

Daarnaast brengt het gebruik van GNS3 in een klasomgeving extra uitdagingen met zich mee. Door de grote flexibiliteit van het platform bestaat het risico dat studenten verdwalen in de vele opties, of dat kleine configuratiefouten tot grote problemen leiden. Dit vraagt om duidelijke structuur en begeleiding vanuit de docent. In de meeste gevallen is het aangewezen om uitgewerkte labhandleidingen en vooraf ingestelde topologie-bestanden aan te bieden, zodat studenten niet volledig vanaf nul hoeven te beginnen, tenzij dit expliciet als leerdoel is voorzien. Het voorbereiden van dergelijke labs vergt echter extra tijd en technische afstemming van de lesgever ~\autocite{gns3_docs2025}.

\vspace{0.3cm}

Een bijkomend aandachtspunt bij het gebruik van GNS3 in onderwijscontexten is het gebrek aan ingebouwde ondersteuning voor multi-user samenwerking en automatische evaluatie. Waar Cisco Packet Tracer via het NetAcad-platform functies aanbiedt voor het automatisch beoordelen van opdrachten en gestructureerde samenwerking tussen studenten, ontbreekt deze functionaliteit bij GNS3. Hoewel het technisch mogelijk is om meerdere clients met een centrale GNS3-server te verbinden, vereist dit een complexe installatie en biedt het geen native ondersteuning voor simultane samenwerking aan hetzelfde project of geautomatiseerde feedbackmechanismen. Dit betekent dat docenten alternatieve evaluatiemethoden moeten hanteren, zoals het laten indienen van configuratiebestanden of het beoordelen van live uitgevoerde opdrachten ~\autocite{gns3_multiuser}.

\vspace{0.3cm}

Tot slot is er de leercurve: omdat GNS3 veel mogelijkheden biedt, kan het voor beginners lastig zijn om ermee te starten. Studenten moeten niet alleen netwerktheorie leren, maar ook leren hoe ze met GNS3 moeten werken, zoals apparaten toevoegen, interfaces verbinden en images instellen. Zonder een goede uitleg of begeleiding kan dit al snel verwarrend of moeilijk worden, zeker in vergelijking met de eenvoudige drag-and-dropomgeving van Packet Tracer \autocite{Amrizal2022}.

\vspace{0.3cm}

Daarom is het belangrijk dat docenten in hun lessen tijd voorzien om studenten stap voor stap wegwijs te maken in het gebruik van GNS3. Er zijn gelukkig veel tutorials beschikbaar en er is een grote online community, maar het blijft de taak van de opleiding om GNS3 op een goede en haalbare manier in te voeren, zodat het bruikbaar blijft voor studenten.


\section{\IfLanguageName{dutch}{Alternatieve tools}{Alternatieve tools
}}
\label{sec:Alternatieve tools: EVE-NG, Cisco VIRL en andere}

Naast Packet Tracer en GNS3 bestaan er nog andere netwerkemulators. Twee relevante alternatieven zijn EVE-NG en Cisco VIRL/CML. Ook Huawei eNSP is een voorbeeld van een simulator die gericht is op één bepaald merk of type netwerkapparatuur.

\vspace{0.3cm}

EVE-NG (Emulated Virtual Environment – Next Generation) lijkt qua werking sterk op GNS3. Het ondersteunt meerdere netwerkleveranciers zoals Cisco, Juniper en Palo Alto en functioneert volledig via de browser. Studenten kunnen hun labs uitvoeren zonder software lokaal te installeren. Dit is vooral interessant voor scholen of trainingsomgevingen die werken met centrale servers en gedeelde toegang. EVE-NG bestaat in een gratis Community Edition en een betaalde Pro-versie met extra functies, zoals ondersteuning voor meerdere gebruikers tegelijk. Volgens gebruikers is GNS3 iets eenvoudiger voor beginners, terwijl EVE-NG meer mogelijkheden biedt, maar ook meer voorbereiding en technische kennis vraagt. Omdat alles via de server draait, is er meestal ook krachtigere hardware nodig dan bij GNS3\autocite{eve-ng}.

\vspace{0.3cm}

Cisco VIRL (nu Cisco CML) is de officiële netwerkemulator van Cisco. Het programma wordt geleverd met een licentie en officiële IOS-images, waardoor gebruikers geen eigen images moeten zoeken zoals bij GNS3. VIRL is vooral bedoeld voor bedrijven en netwerkprofessionals en is niet gratis. Soms zijn er proeflicenties of educatieve versies beschikbaar, maar voor standaardgebruik in het onderwijs is het minder interessant door de kostprijs en complexiteit. Voor grote labs of situaties waarin officiële Cisco-software vereist is, kan het wel een nuttige oplossing zijn \autocite{cisco-cml}.

\vspace{0.3cm}

Huawei eNSP is een simulator voor Huawei-netwerkapparatuur. Het werkt vergelijkbaar met Packet Tracer, maar is gericht op Huawei-certificeringen. Voor HOGENT is dit momenteel minder relevant, tenzij er in de toekomst met Huawei-apparatuur gewerkt zou worden \autocite{huawei-ensp}.

\vspace{0.3cm}

Andere tools zoals Mininet (voor SDN-onderzoek) en Boson NetSim (een commerciële CCNA-simulator) bestaan ook, maar worden weinig gebruikt in onze context. Mininet is vooral geschikt voor onderzoeksdoeleinden. Boson NetSim is betalend, populair bij zelfstudie, maar biedt minder uitgewerkte scenario’s dan Packet Trace \autocite{huawei-ensp, cisco-cml}.


