%%=============================================================================
%% Inleiding
%%=============================================================================

\chapter{\IfLanguageName{dutch}{Inleiding}{Introduction}}%
\label{ch:inleiding}

Binnen het domein van systeem- en netwerkbeheer is het van belang dat studenten niet alleen theoretische kennis verwerven, maar ook praktische ervaring opdoen in het configureren en beheren van netwerken. Aangezien fysieke netwerkapparatuur vaak duur en niet altijd beschikbaar is, maken opleidingen steeds vaker gebruik van simulatie- en emulatiesoftware. Deze tools bieden studenten de mogelijkheid om netwerkconcepten op een toegankelijke en praktijkgerichte manier aan te leren, zonder dat daar meteen dure hardware voor nodig is ~\autocite{Gomez2023}.

\vspace{0.3cm}

Zo is Packet Tracer, een toegankelijke simulator van Cisco, uitgegroeid tot een wereldwijd gebruikte tool binnen het Networking Academy-programma. Miljoenen studenten maken er gebruik van om netwerkvaardigheden in te oefenen zonder fysieke apparatuur ~\autocite{ciscoP}. Uit de ervaring van netwerkopleiders blijkt dat herhaaldelijk oefenen in virtuele labo-omgevingen belangrijk is om netwerkconcepten te begrijpen en te onthouden ~\autocite{Maes2022}.

\vspace{0.5cm}

Naast Packet Tracer bestaan er geavanceerdere tools zoals Graphical Network Simulator 3 (GNS3), een open-source emulatieplatform dat toelaat om echte router- en switchsoftware te draaien in virtuele netwerken. Beide tools hebben hun eigen sterktes en zwaktes~\autocite{gns3_doc2025}.

\vspace{0.3cm}

In deze bachelorproef wordt onderzocht in welke mate GNS3 en Cisco Packet Tracer van elkaar verschillen in functionaliteit en geschiktheid voor verschillende opleidings- en trainingsniveaus in netwerkconfiguratie en simulatie.



\section{\IfLanguageName{dutch}{Probleemstelling}{Problem Statement}}%
\label{sec:probleemstelling}

De meeste hogescholen en universiteiten maken binnen netwerkgerichte opleidingen gebruik van Cisco Packet Tracer, vooral in het kader van de CCNA-leerstof. Ook aan Hogeschool Gent wordt deze tool gebruikt binnen de opleiding Systeem- en Netwerkbeheer. Packet Tracer is gebruiksvriendelijk en laagdrempelig, maar blijft beperkt tot de functionaliteiten die Cisco aanbiedt en simuleert niet alle aspecten van echte netwerkhardware. Tegelijk wordt ook GNS3 gebruikt, zowel in het hoger onderwijs als in de professionele sector, vanwege de realistische emulatie van netwerkapparatuur en de grotere flexibiliteit in netwerkomgevingen~\autocite{gns3_docs2025}.

\vspace{0.5cm}

Het probleem is dat het onduidelijk is welke tool het meest geschikt is voor welk opleidingsniveau. Beginnende studenten leren mogelijk sneller met een eenvoudige simulator, terwijl gevorderde studenten beter leren in een omgeving die sterk lijkt op echte netwerksituaties. Daarnaast stelt zich de vraag of het voor een onderwijsinstelling zinvol is om over te stappen van het vertrouwde Packet Tracer naar GNS3, gezien de mogelijke impact op leerresultaten, technische haalbaarheid en het onderhoud van de leeromgeving.

\section{\IfLanguageName{dutch}{Onderzoeksvraag}{Research question}}%
\label{sec:onderzoeksvraag}
Om inzicht te krijgen in dit onderzoek, wordt de volgende hoofdvraag geformuleerd. 



\begin{itemize}
   \item In welke mate verschillen GNS3 en Cisco Packet Tracer van elkaar op vlak van functionaliteiten en geschiktheid voor verschillende niveaus van netwerkonderwijs en simulatie?
\end{itemize}

Om deze brede vraag te kunnen beantwoorden, wordt ze opgedeeld in enkele specifieke deelvragen:

\begin{itemize}
    \item \textbf{Functionele vergelijking:} Welke functionaliteiten en kenmerken bieden Cisco Packet Tracer en GNS3, en welke belangrijke verschillen bestaan er (bv. ondersteunde protocollen, apparaattypes, realisme van simulatie)?
    
    \item \textbf{Doelgroep en gebruiksgemak:} In welke mate is elke tool gebruiksvriendelijk en wat is de leercurve voor studenten op verschillende opleidingsniveaus (beginnend netwerkonderwijs vs. gevorderde opleidingen zoals CCNA)?
    
    \item \textbf{Praktische haalbaarheid:} Welke moeilijkheden of uitdagingen komen naar voren bij het gebruiken van beide tools in concrete netwerkscenario’s (bv. installatiecomplexiteit, benodigde hardwarebronnen, compatibiliteit met echte apparatuur of bestaande labs)?
    
    \item \textbf{Aanbeveling onderwijs} Is het binnen een onderwijscontext zoals HoGent aan te raden om (volledig of gedeeltelijk) over te stappen van Cisco Packet Tracer naar GNS3, rekening houdend met de voor- en nadelen voor lesgevers en studenten?
    

\end{itemize}
\section{\IfLanguageName{dutch}{Onderzoeksdoelstelling}{Research objective}}%
\label{sec:onderzoeksdoelstelling}

De doelstelling van dit onderzoek is om een onderbouwd inzicht te bieden in de voor- en nadelen van Cisco Packet Tracer en GNS3 binnen het onderwijs. We beperken ons tot een vergelijking van deze twee specifieke tools, aangezien het de twee meest gebruikte oplossingen zijn in de context van Cisco-gebaseerde netwerkopleidingen. Andere netwerkemulators (zoals Cisco’s eigen CML/VIRL of EVE-NG) vallen buiten scope, maar worden kort vermeld ter context indien relevant. 

\vspace{0.3cm}

De focus ligt op educatieve toepassingen tot op het niveau van CCNA (Cisco Certified Network Associate), omdat dit aansluit bij de curriculumonderdelen aan HoGent. Complexere enterprise-netwerksimulaties of certificeringsniveaus hoger dan CCNA worden niet uitvoerig behandeld, behalve wanneer dit nodig is om de beperkingen van beide tools te duiden.

\section{\IfLanguageName{dutch}{Opzet van deze bachelorproef}{Structure of this bachelor thesis}}%
\label{sec:opzet-bachelorproef}

% Het is gebruikelijk aan het einde van de inleiding een overzicht te
% geven van de opbouw van de rest van de tekst. Deze sectie bevat al een aanzet
% die je kan aanvullen/aanpassen in functie van je eigen tekst.

De rest van deze bachelorproef is als volgt opgebouwd:

In Hoofdstuk~\ref{ch:stand-van-zaken} wordt een overzicht gegeven van de stand van zaken binnen het onderzoeksdomein, op basis van een literatuurstudie.

\vspace{0.3cm}

In Hoofdstuk~\ref{ch:methodologie} wordt de methodologie toegelicht en worden de gebruikte onderzoekstechnieken besproken om een antwoord te kunnen formuleren op de onderzoeksvragen.

\vspace{0.3cm}

% TODO: Vul hier aan voor je eigen hoofstukken, één of twee zinnen per hoofdstuk

In Hoofdstuk~\ref{ch:conclusie}, tenslotte, wordt de conclusie gegeven en een antwoord geformuleerd op de onderzoeksvragen. Daarbij wordt ook een aanzet gegeven voor toekomstig onderzoek binnen dit domein.