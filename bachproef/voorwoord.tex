%%=============================================================================
%% Voorwoord
%%=============================================================================

\chapter*{\IfLanguageName{dutch}{Woord vooraf}{Preface}}%
\label{ch:voorwoord}

%% TODO:
%% Het voorwoord is het enige deel van de bachelorproef waar je vanuit je
%% eigen standpunt (``ik-vorm'') mag schrijven. Je kan hier bv. motiveren
%% waarom jij het onderwerp wil bespreken.
%% Vergeet ook niet te bedanken wie je geholpen/gesteund/... heeft

Deze bachelorproef werd geschreven in het kader van het behalen van het diploma Bachelor in de Toegepaste Informatica, afstudeerrichting Systeem- en Netwerkbeheer. Tijdens het traject werd het oorspronkelijke onderzoeksonderwerp herzien. Na overleg met mijn promotor werd duidelijk dat het oorspronkelijke onderwerp onvoldoende aansloot bij de praktische haalbaarheid en academische relevantie. Daarom werd in samenspraak besloten om het onderzoek te verleggen naar een vergelijkende studie tussen Cisco Packet Tracer en GNS3. Dit nieuwe onderzoek sloot beter aan bij mijn opleiding, persoonlijke interesse en de beschikbare middelen, en vormde het uitgangspunt voor de bachelorproef die hier wordt uitgewerkt.

\vspace{0.3cm}

Dit eindwerk had ik niet kunnen realiseren zonder de steun en inzet van een aantal mensen, aan wie ik mijn oprechte dank wil uitspreken.

\vspace{0.3cm}

Mijn bijzondere dank gaat uit naar mijn promotor, de heer Lieven Smits. Zijn begeleiding, feedback en waardevolle adviezen hebben me geholpen om tot een sterk en relevant onderzoeksvoorstel te komen. Vooral tijdens de overgang naar een nieuw onderwerp heeft zijn bijdrage een belangrijke rol gespeeld in de goede afronding van deze bachelorproef.

\vspace{0.3cm}

Mijn dank gaat ook uit naar mijn co-promotor, de heer Stijn Boussemaere, die bereid was het nieuwe onderwerp mee te ondersteunen. Zijn openheid, flexibiliteit en feedback hebben een belangrijke bijdrage geleverd aan het slagen van deze bachelorproef.