%%=============================================================================
%% Samenvatting
%%=============================================================================

% TODO: De "abstract" of samenvatting is een kernachtige (~ 1 blz. voor een
% thesis) synthese van het document.
%
% Een goede abstract biedt een kernachtig antwoord op volgende vragen:
%
% 1. Waarover gaat de bachelorproef?
% 2. Waarom heb je er over geschreven?
% 3. Hoe heb je het onderzoek uitgevoerd?
% 4. Wat waren de resultaten? Wat blijkt uit je onderzoek?
% 5. Wat betekenen je resultaten? Wat is de relevantie voor het werkveld?
%
% Daarom bestaat een abstract uit volgende componenten:
%
% - inleiding + kaderen thema
% - probleemstelling
% - (centrale) onderzoeksvraag
% - onderzoeksdoelstelling
% - methodologie
% - resultaten (beperk tot de belangrijkste, relevant voor de onderzoeksvraag)
% - conclusies, aanbevelingen, beperkingen
%
% LET OP! Een samenvatting is GEEN voorwoord!

%%---------- Nederlandse samenvatting -----------------------------------------
%
% TODO: Als je je bachelorproef in het Engels schrijft, moet je eerst een
% Nederlandse samenvatting invoegen. Haal daarvoor onderstaande code uit
% commentaar.
% Wie zijn bachelorproef in het Nederlands schrijft, kan dit negeren, de inhoud
% wordt niet in het document ingevoegd.

\IfLanguageName{english}{%
\selectlanguage{dutch}
\chapter*{Samenvatting}
\lipsum[1-4]
\selectlanguage{english}
}{}

%%---------- Samenvatting -----------------------------------------------------
% De samenvatting in de hoofdtaal van het document

\chapter*{\IfLanguageName{dutch}{Samenvatting}{Abstract}}

In de opleiding netwerken wordt steeds meer ingezet op praktijkgericht leren. Simulatie- en emulatietools vormen daarbij een toegankelijk alternatief voor fysieke netwerkapparatuur. Deze bachelorproef vergelijkt twee veelgebruikte tools: Cisco Packet Tracer, dat netwerkgedrag simuleert via eenvoudige modellen, en GNS3, dat echte netwerksoftware uitvoert in een virtuele omgeving.

\vspace{0.3cm}

Het onderzoek gaat na welke tool het meest geschikt is voor netwerkopleidingen in het hoger onderwijs, afhankelijk van het opleidingsniveau en de leerdoelen. Eerst wordt het verschil tussen simulatie en emulatie toegelicht. Vervolgens komen de kenmerken van beide tools aan bod, zoals Gebruiksvriendelijkheid, functionaliteit, uitbreidbaarheid en beperkingen.

\vspace{0.3cm}

De vergelijking wordt ondersteund door drie uitgewerkte scenario’s met oplopende moeilijkheidsgraad, die in zowel Packet Tracer als GNS3 worden getest. Daarbij wordt gekeken naar configuratiemogelijkheden, foutopsporing, systeembelasting en het leereffect voor studenten.

\vspace{0.3cm}

Op basis van deze analyse worden aanbevelingen geformuleerd voor onderwijsinstellingen over het inzetten van deze tools, afgestemd op de technische context en het niveau van de studenten.

