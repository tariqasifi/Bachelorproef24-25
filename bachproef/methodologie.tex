%%=============================================================================
%% Methodologie
%%=============================================================================

\chapter{\IfLanguageName{dutch}{Methodologie}{Methodology}}%
\label{ch:methodologie}
Dit onderzoek volgt een gestructureerde aanpak, verdeeld in vier opeenvolgende fasen om systematisch inzicht te verkrijgen in Man-in-the-Middle (MitM)-aanvallen en de meest geschikte beveiligingsmaatregelen voor diverse netwerkomgevingen.

In de eerste fase voeren we een grondig literatuuronderzoek uit naar verschillende soorten MitM-aanvallen en de beveiligingsmethoden die momenteel beschikbaar zijn. We gebruiken wetenschappelijke artikelen, rapporten en casestudies om inzicht te krijgen in aanvalstechnieken zoals ARP-spoofing, DNS-spoofing en SSL-stripping. Dit onderzoek leidt tot een overzicht van aanvalstypen en mogelijke verdedigingsstrategieën en vormt de theoretische basis voor verdere analyse en praktische toepassing.


Op basis van de literatuur selecteren we relevante beveiligingsmaatregelen, zoals encryptie, authenticatie en detectiesystemen, die effectief kunnen zijn tegen MitM-aanvallen. Elke maatregel wordt beoordeeld op haar theoretische effectiviteit, haalbaarheid en geschiktheid voor specifieke netwerkomgevingen, waaronder IoT-apparaten. Deze analyse helpt ons om maatregelen te selecteren voor praktische evaluatie in de volgende fase.

In deze fase ontwikkelen we een Proof of Concept (PoC) voor één of meer geselecteerde beveiligingsmaatregelen en testen deze in een gesimuleerde netwerkomgeving. Hierin voeren we MitM-aanvalsscenario's uit om de effectiviteit en prestaties van de maatregelen te evalueren. De focus ligt op meetbare resultaten zoals gegevensintegriteit, vertrouwelijkheid en eventuele impact op netwerkprestaties. Deze testen bieden praktische inzichten in de haalbaarheid en kosten van de maatregelen.


Op basis van de testresultaten worden de beveiligingsmaatregelen geëvalueerd. Er wordt een implementatieadvies opgesteld voor organisaties, met aanbevelingen over de inzet van MitM-beveiliging in verschillende netwerkconfiguraties, inclusief IoT-omgevingen. Dit advies bevat praktische overwegingen voor schaalbaarheid en kosten, zodat organisaties goed geïnformeerde beslissingen kunnen nemen.

%% TODO: In dit hoofstuk geef je een korte toelichting over hoe je te werk bent
%% gegaan. Verdeel je onderzoek in grote fasen, en licht in elke fase toe wat
%% de doelstelling was, welke deliverables daar uit gekomen zijn, en welke
%% onderzoeksmethoden je daarbij toegepast hebt. Verantwoord waarom je
%% op deze manier te werk gegaan bent.
%% 
%% Voorbeelden van zulke fasen zijn: literatuurstudie, opstellen van een
%% requirements-analyse, opstellen long-list (bij vergelijkende studie),
%% selectie van geschikte tools (bij vergelijkende studie, "short-list"),
%% opzetten testopstelling/PoC, uitvoeren testen en verzamelen
%% van resultaten, analyse van resultaten, ...
%%
%% !!!!! LET OP !!!!!
%%
%% Het is uitdrukkelijk NIET de bedoeling dat je het grootste deel van de corpus
%% van je bachelorproef in dit hoofstuk verwerkt! Dit hoofdstuk is eerder een
%% kort overzicht van je plan van aanpak.
%%
%% Maak voor elke fase (behalve het literatuuronderzoek) een NIEUW HOOFDSTUK aan
%% en geef het een gepaste titel.



